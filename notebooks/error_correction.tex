<h1>Quantum Error Correction</h1>

\begin{verbatim}
%load_ext sympyprinting
\end{verbatim}

\begin{verbatim}
from sympy import sqrt, symbols, Rational
from sympy import expand, Eq, Symbol, simplify, exp, sin
from sympy.physics.quantum import *
from sympy.physics.quantum.qubit import *
from sympy.physics.quantum.gate import *
from sympy.physics.quantum.grover import *
from sympy.physics.quantum.qft import QFT, IQFT, Fourier
from sympy.physics.quantum.circuitplot import circuit_plot
\end{verbatim}

<h2>5 qubit code</h2>

\begin{verbatim}
M0 = Z(1)*X(2)*X(3)*Z(4); M0

\end{verbatim}

\begin{verbatim}
M1 = Z(2)*X(3)*X(4)*Z(0); M1

\end{verbatim}

\begin{verbatim}
M2 = Z(3)*X(4)*X(0)*Z(1); M2

\end{verbatim}

\begin{verbatim}
M3 = Z(4)*X(0)*X(1)*Z(2); M3

\end{verbatim}

These operators should mutually commute.

\begin{verbatim}
gate_simp(Commutator(M0,M1).doit())

\end{verbatim}

And square to the identity.

\begin{verbatim}
for o in [M0,M1,M2,M3]:
    display(gate_simp(o*o))

\end{verbatim}

<h2>Codewords</h2>

\begin{verbatim}
zero = Rational(1,4)*(1+M0)*(1+M1)*(1+M2)*(1+M3)*IntQubit(0, 5); zero

\end{verbatim}

\begin{verbatim}
qapply(4*zero)

\end{verbatim}

\begin{verbatim}
one = Rational(1,4)*(1+M0)*(1+M1)*(1+M2)*(1+M3)*IntQubit(2**5-1, 5); one

\end{verbatim}

\begin{verbatim}
qapply(4*one)

\end{verbatim}

<h2>The encoding circuit</h2>

\begin{verbatim}
encoding_circuit = H(3)*H(4)*CNOT(2,0)*CNOT(3,0)*CNOT(4,0)*H(1)*H(4)*\
                   CNOT(2,1)*CNOT(4,1)*H(2)*CNOT(3,2)*CNOT(4,2)*H(3)*\
                   H(4)*CNOT(4, 3)*Z(4)*H(4)*Z(4)

\end{verbatim}

\begin{verbatim}
circuit_plot(encoding_circuit, nqubits=5, scale=0.5)
\end{verbatim}

\begin{verbatim}
represent(4*encoding_circuit, nqubits=5)
\end{verbatim}

\begin{verbatim}
%notebook save qerror.ipynb
\end{verbatim}

\begin{verbatim}
%notebook load decompose.ipynb
\end{verbatim}

