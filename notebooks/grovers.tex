<h1>Grover's Algorithm</h1>

\begin{verbatim}
%load_ext sympyprinting
\end{verbatim}

\begin{verbatim}
from sympy import sqrt, symbols, Rational
from sympy import expand, Eq, Symbol, simplify, exp, sin
from sympy.physics.quantum import *
from sympy.physics.quantum.qubit import *
from sympy.physics.quantum.gate import *
from sympy.physics.quantum.grover import *
from sympy.physics.quantum.qft import QFT, IQFT, Fourier
from sympy.physics.quantum.circuitplot import circuit_plot
\end{verbatim}

\begin{verbatim}
nqubits = 3
\end{verbatim}

Grover's algorithm is a quantum algorithm which searches an unordered database
(inverts a function).<br/> Provides polynomial speedup over classical brute-
force search ($O(\sqrt{N}) vs. O(N))$

Define a black box function that returns True if it is passed the state we are
searching for.

\begin{verbatim}
def black_box(qubits):
    return True if qubits == IntQubit(1, qubits.nqubits) else False
\end{verbatim}

Build a uniform superposition state to start the search.

\begin{verbatim}
psi = superposition_basis(nqubits); psi
\end{verbatim}

\begin{verbatim}
v = OracleGate(nqubits, black_box)
\end{verbatim}

Perform two iterations of Grover's algorithm.  Each iteration, the amplitude of
the target state increases.

\begin{verbatim}
iter1 = qapply(grover_iteration(psi, v)); iter1
\end{verbatim}

\begin{verbatim}
iter2 = qapply(grover_iteration(iter1, v)); iter2
\end{verbatim}

A single shot measurement is performed to retrieve the target state.

\begin{verbatim}
measure_all_oneshot(iter2)
\end{verbatim}

\begin{verbatim}
%notebook save grovers.ipynb
\end{verbatim}

\begin{verbatim}
%notebook load qft.ipynb
\end{verbatim}

