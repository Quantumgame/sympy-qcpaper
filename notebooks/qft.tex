<h1>Quantum Fourier Transform</h1>

\begin{verbatim}
%load_ext sympyprinting
\end{verbatim}

\begin{verbatim}
from sympy import sqrt, symbols, Rational
from sympy import expand, Eq, Symbol, simplify, exp, sin
from sympy.physics.quantum import *
from sympy.physics.quantum.qubit import *
from sympy.physics.quantum.gate import *
from sympy.physics.quantum.grover import *
from sympy.physics.quantum.qft import QFT, IQFT, Fourier
from sympy.physics.quantum.circuitplot import circuit_plot
\end{verbatim}

QFT is useful for a quantum algorithm for factoring numbers which is
exponentially faster than what is thought to be possible on a classical
machine.<br> The transform does a DFT on the state of a quantum system<br> There
is a simple decomposition of the QFT in terms of a few elementary gates.

<h2>QFT Gate and Circuit</h2>

Build a 3 qubit QFT and decompose it into primitive gates.

\begin{verbatim}
fourier = QFT(0,3).decompose(); fourier
\end{verbatim}

\begin{verbatim}
circuit_plot(fourier, nqubits=3)
\end{verbatim}

The QFT circuit can be represented in various symbolic forms.

\begin{verbatim}
m = represent(fourier, nqubits=3)
\end{verbatim}

\begin{verbatim}
m
\end{verbatim}

\begin{verbatim}
represent(Fourier(0,3), nqubits=3)*4/sqrt(2)
\end{verbatim}

<h2>QFT in Action</h2>

Build a 3 qubit state to take the QFT of.

\begin{verbatim}
state = (Qubit('000') + Qubit('010') + Qubit('100') + Qubit('110'))/sqrt(4); state
\end{verbatim}

Perform the QFT.

\begin{verbatim}
qapply(fourier*state)
\end{verbatim}

\begin{verbatim}
%notebook save qft.ipynb
\end{verbatim}

\begin{verbatim}
%notebook load density_matrix.ipynb
\end{verbatim}

