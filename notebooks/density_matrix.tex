<h1>Symbolic Density Matrices</h1>

\begin{verbatim}
%load_ext sympyprinting
\end{verbatim}

\begin{verbatim}
from sympy import sqrt, symbols, Rational, S
from sympy import expand, Eq, Symbol, simplify, exp, sin
from sympy.physics.quantum import *
from sympy.physics.quantum.qubit import *
from sympy.physics.quantum.gate import *
from sympy.physics.quantum.grover import *
from sympy.physics.quantum.qft import QFT, IQFT, Fourier
from sympy.physics.quantum.circuitplot import circuit_plot
from sympy.physics.quantum.densityOp import *
\end{verbatim}

Density Operators are used to represent systems which has both classical
mixtures as well as quantum superpositions.

\begin{verbatim}
state = Density([Qubit('00'),0.5],[(Qubit('00')+Qubit('11'))/sqrt(2),0.5]); state
\end{verbatim}

In addition to Dirac notation, code can represent the state as a matrix.

\begin{verbatim}
represent(state, nqubits=2)
\end{verbatim}

Gate operators can be applied to density operators.

\begin{verbatim}
qapply(state.operate_on(HadamardGate(0)))
\end{verbatim}

Von Neumann Entropy can be calculated.

\begin{verbatim}
state.entropy(nqubits=2)
\end{verbatim}

\begin{verbatim}
Density([Qubit('00'),1]).entropy(nqubits=2)
\end{verbatim}

We can also create symbolic states.

\begin{verbatim}
genState = Density([Ket('psi'),.5],[Ket('phi'),.5]); genState
\end{verbatim}

\begin{verbatim}
%notebook save density_matrix.ipynb
\end{verbatim}

\begin{verbatim}
%notebook load qerror.ipynb
\end{verbatim}

